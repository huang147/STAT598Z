\def\problemset#1#2#3{
\noindent\rule{16.5cm}{1pt}
\begin{center}
  \parbox{16.5cm}{\bf
    STAT 598Z Homework 1 \\
    Instructor: Prof. S V N Vishwanathan \hfill Jiajie Huang\\
    Due Jan 29, 2013 \hfill huang147@purdue.edu
    }
\end{center}
\noindent\rule{16.5cm}{0.5pt}
}

\newcommand{\lb}[1]{\left \lfloor #1 \right \rfloor}
\newcommand{\bmat}[1]{\begin{bmatrix} #1 \end{bmatrix}}
\documentclass[fleqn, 11pt]{article}
\usepackage{fullpage}
\usepackage{hyperref}
\usepackage{ulem}
\usepackage{amsmath}
\usepackage{algorithm}
%\usepackage{algorithmic}
\usepackage{algpseudocode}
\setlength{\parindent}{0in}
\usepackage{graphics}
\usepackage{graphicx}
\usepackage{mathtools}


\begin{document}

\problemset{3}{Problem Set 2}{\today}


\section*{Problem 1}
\subsection*{(a) The count of row number}
\begin{verbatim}
Code: 
wc -l dummy.txt 

Output: 
4 dummy.txt
\end{verbatim}

\subsection*{(b) The count of column number}
\begin{verbatim}
Code: 
head -n 1 dummy.txt | wc -w

Output: 
4
\end{verbatim}

\subsection*{(c) The count of total entries}
\begin{verbatim}
Code: 
wc -w dummy.txt

Output: 
16 dummy.txt
\end{verbatim}

\subsection*{(d) Display the element $a_{11}$}
\begin{verbatim}
Code: 
cut -d " " -f 1 dummy.txt | head -n 1

Output: 
1
\end{verbatim}

\subsection*{(e) Display the element $a_{nn}$}
\begin{verbatim}
Code: 
tail -n 1 dummy.txt | cut -f `tail -n 1 dummy.txt | wc -w` -d ' '

Output: 
16
\end{verbatim}







\end{document}